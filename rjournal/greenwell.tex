% !TeX root = RJwrapper.tex
\title{Explaining Predictions with Shapley Values---An Introduction to the
fastshap Package}
\author{by Brandon M. Greenwell}

\maketitle

\abstract{%
An abstract of less than 150 words.
}

\hypertarget{introduction}{%
\subsection{Introduction}\label{introduction}}

Introductory section which may include references in parentheses
\citep{R}, or cite a reference such as \citet{R} in the text.

\hypertarget{background}{%
\subsection{Background}\label{background}}

This section may contain a figure such as Figure \ref{fig:Rlogo}.

\begin{Schunk}
\begin{figure}[htbp]

{\centering \includegraphics[width=2in]{/Users/b780620/Desktop/devel/fastshap/man/figures/logo} 

}

\caption[The logo of fastshap]{The logo of fastshap.}\label{fig:Rlogo}
\end{figure}
\end{Schunk}

\hypertarget{estimating-shapley-values-in-practice}{%
\subsubsection{Estimating Shapley values in
practice}\label{estimating-shapley-values-in-practice}}

TBD.

\begin{algorithm}
\begin{enumerate}
  \item For $i = 1, 2, \dots, j$:
  \begin{enumerate}
    \item Permute the values of feature $X_i$ in the training data.
    \item Recompute the performance metric on the permuted data $\mathcal{M}_{perm}$.
    \item Record the difference from baseline using $imp\left(X_i\right) = \mathcal{M}_{perm} - \mathcal{M}_{orig}$.
  \end{enumerate}
  \item Return the VI scores $imp\left(X_1\right), imp\left(X_2\right), \dots, imp\left(X_j\right)$.
\end{enumerate}
\caption{Monte Carlo algorithm for approximating Shapley values. \label{alg:permute}}
\end{algorithm}

\hypertarget{special-cases}{%
\subsubsection{Special cases}\label{special-cases}}

TBD.

\hypertarget{linear-models-linearshap}{%
\paragraph{Linear models: LinearSHAP}\label{linear-models-linearshap}}

TBD.

\hypertarget{tree-based-models-treeshap}{%
\paragraph{Tree-based models:
TreeSHAP}\label{tree-based-models-treeshap}}

TBD.

\hypertarget{shapley-values-in-r}{%
\subsection{Shapley values in R}\label{shapley-values-in-r}}

TBD.

\hypertarget{example-predicing-sales-prices}{%
\subsection{Example: predicing sales
prices}\label{example-predicing-sales-prices}}

TBD.

\hypertarget{example-default-of-credit-card-clients}{%
\subsection{Example: default of credit card
clients}\label{example-default-of-credit-card-clients}}

TBD.

\hypertarget{summary}{%
\subsection{Summary}\label{summary}}

This file is only a basic article template. For full details of
\emph{The R Journal} style and information on how to prepare your
article for submission, see the
\href{https://journal.r-project.org/share/author-guide.pdf}{Instructions
for Authors}.

\bibliography{greenwell}


\address{%
Brandon M. Greenwell\\
University of Cincinnati\\
2925 Campus Green Dr\\ Cincinnati, OH 45221\\ United States of America\\ ORCiD---\href{https://orcid.org/0000-0002-8120-0084}{0000-0002-8120-0084}\\
}
\href{mailto:greenwell.brandon@gmail.com}{\nolinkurl{greenwell.brandon@gmail.com}}

